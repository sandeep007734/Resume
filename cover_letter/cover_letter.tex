  \documentclass[10pt, a4paper]{article}
  \usepackage[left=2cm, top=1.5cm, right=2cm, bottom=1.8cm]{geometry}
  \usepackage{xcolor}
  \usepackage{hyperref}
  \usepackage{soul}

  % Sans-serif font
  % \renewcommand{\familydefault}{\sfdefault}

  % Remove page numbers
  \pagestyle{empty}

  % Hyperlink setups
  \hypersetup{
      colorlinks=true,
      linkcolor=black,
      urlcolor=blue,
      pdfborder={0 0 0}
  }

  \begin{document}

  % Slightly tighter spacing to fit one page
  \setlength{\parskip}{0pt}
  \setlength{\parindent}{0pt}
  \setlength{\itemsep}{0.2em}
  \setlength{\topsep}{0.2em}
  \setlength{\partopsep}{0pt}

  % Header
  \begin{center}
  {\LARGE\bfseries Dr. Sandeep Kumar}\\[0.3em]
  {\large Research Scientist @ Intel, Bengaluru, India}\\[0.5em]
  \href{mailto:sandeep007734@gmail.com}{sandeep007734@gmail.com} | (+91) 8277361995\\
  \href{https://sandeep007734.github.io}{sandeep007734.github.io} | 
  \href{https://github.com/sandeep007734}{github.com/sandeep007734} | 
  \href{https://linkedin.com/in/sandeep007734}{linkedin.com/in/sandeep007734}
  \end{center}

  \vspace{0.6em}

  % Letter details
  \noindent\today

  % \vspace{1em}

  % \noindent Hiring Manager\\
  % Example Labs\\
  % 123 Innovation Way\\
  % Bengaluru, KA

  % \vspace{1em}

  % \noindent\textbf{Re: Application for Research Scientist}

  % \vspace{1em}

  % \noindent Dear Hiring Manager,

  \vspace{0.6em}

% \noindent\textbf{Cover Letter}


\section*{Cover Letter}

\noindent
\textbf{Why I'm excited about this role}\\
I am excited by the engineering challenges at the intersection of large-scale model training/inference and high-performance systems—especially making multimodal LLMs fast, reliable, and cost-efficient on highly interconnected GPU clusters. I enjoy work that requires tight feedback loops between profiling, algorithmic choices, and low-level systems optimization.



\vspace{0.6em}
\noindent
\textbf{Relevant experience I bring}\\
I am a systems researcher and performance-focused software engineer at Intel Labs, working on AI/ML efficiency with a particular focus on memory behavior (memory tiering for LLMs and RAG systems) and production-grade systems work (including upstream Linux kernel memory-management contributions). I regularly profile and debug across the stack (OS/kernel, drivers/accelerators, userspace) and use telemetry to translate bottlenecks into targeted optimizations.

I have hands-on experience integrating and tuning hardware accelerators (Intel IAA/DSA) for real workloads, and I am comfortable making kernel- and driver-adjacent changes when those are the source of performance or correctness issues. During my PhD, I also worked on low-level systems around Intel SGX (e.g., enclave performance/benchmarking and OS-facing mechanisms), which strengthened my ability to debug across privilege boundaries.

I build and optimize Python-based AI workflows (including PyTorch training/inference) and go deeper, when needed, into native code and systems configuration. For DNN training, I have worked directly with PyTorch-based training code, including profiling/instrumentation and targeted changes to reduce memory pressure and improve efficiency. I am comfortable reasoning about distributed execution fundamentals and excited to apply that to GPU-centric stacks (e.g., PyTorch Distributed with NCCL) and modern large-scale training/inference frameworks.

\vspace{0.6em}
\noindent
\textbf{How I would contribute in your stack}\\
\begin{itemize}
  \item \textbf{Scale-out training:} Improve parallel efficiency by profiling communication/memory bottlenecks and validating scaling strategies with rigorous measurement.
  \item \textbf{High-throughput inference:} Tune multimodal LLM serving for throughput and tail latency, leveraging my experience optimizing LLM inference and RAG systems.
  \item \textbf{Deep performance work:} Debug and optimize low-level behavior (kernels, drivers, OS configuration) to remove system bottlenecks.
\end{itemize}

\noindent
I would welcome the opportunity to discuss how my systems performance background in AI infrastructure can help you push the boundaries of what is computationally possible.


  \vspace{1.4em}

  \noindent Sincerely,

  \vspace{1.4em}

  \noindent Dr. Sandeep Kumar

  \vspace{0.5em}

  % \noindent\textit{Enclosure: Curriculum Vitae}

  \end{document}
