%!TEX TS-program = xelatex
%!TEX encoding = UTF-8 Unicode

% Point to the shared fonts folder (evaluated while the class loads)
\makeatletter
\def\@fontdir{../fonts/}
\makeatother

\documentclass[11pt, a4paper]{awesome-cv}
\usepackage{xcolor}

% Page layout
\geometry{left=1.4cm, top=.8cm, right=1.4cm, bottom=1.8cm, footskip=.5cm}

% Fonts are one level up (shared with resume)
\fontdir[../fonts/]

% Mono-color palette (matches resume)
\colorlet{awesome}{awesome-darknight}
\definecolor{awesome}{HTML}{000000}
\definecolor{darktext}{HTML}{000000}
\definecolor{text}{HTML}{000000}
\definecolor{graytext}{HTML}{000000}
\definecolor{lighttext}{HTML}{000000}

% Personal info (header)
\name{Dr. Sandeep}{Kumar}
\position{Research Scientist @ Intel, Bengaluru, India}
\address{--}
\mobile{(+91) 8277361995}
\email{sandeep007734@gmail.com}
\homepage{sandeep007734.github.io}
\github{sandeep007734}
\linkedin{sandeep007734}

% Letter metadata
\recipient{Hiring Manager}{Example Labs\\123 Innovation Way\\Bengaluru, KA}
\letterdate{\today}
\lettertitle{Application for Research Scientist}
\letteropening{Dear Hiring Manager,}
\letterclosing{Sincerely,}
\letterenclosure[Enclosure]{Curriculum Vitae}

\begin{document}
\makecvheader
\makelettertitle

\begin{cvletter}
  \lettersection{Why I'm Excited}
  I thrive at the intersection of systems research and product delivery. At Intel Labs I prototype, evaluate, and iterate quickly, which aligns with Example Labs' pace and appetite for experimentation.

  \lettersection{What I Bring}
  My recent work focused on hardware-software co-design: profiling workloads, tuning memory hierarchies, and shipping tooling that reduces developer friction. I enjoy collaborating across research, firmware, and product teams to turn ideas into measurable wins.

  \lettersection{What Success Looks Like}
  In the first 90 days, I would pair with your platform team to identify one latency-critical path, build a reproducible benchmark, and prototype at least two design options with data-driven trade-offs.
\end{cvletter}

\makeletterclosing
\end{document}
