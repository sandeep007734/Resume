\begin{itemize}



\item \normalsize{\textsc{Scalable Performance Tuning of Hadoop MapReduce: A Noisy Gradient Approach}} \\
\footnotesize{The simultaneous perturbation stochastic approximation (\textbf{SPSA}) algorithm is a noisy gradient algorithm that has been successfully deployed for parameter tuning in a variety of applications ranging from traffic control to service systems. The SPSA algorithm tunes the parameters by directly observing the performance of the real system. Further, the SPSA is independent of parameter dimensions and requires only two or fewer observations per iteration. We tune the Hadoop parameters using the Simultaneous Perturbation Stochastic Approximation (SPSA) algorithm in order to achieve better performance than the default configuration. We are currently comparing this method with other techniques such as Starfish.}\\\phantom{.}\hfill\small Advisor: \href{http://drona.csa.iisc.ernet.in/~gopi/}{Prof. K. \textsc{Gopinath}} \& \href{http://drona.csa.iisc.ernet.in/~shalabh/}{Prof. Shalabh Bhatnagar} 
\\\phantom{.}\hfill\small Collaborators: \href{http://stochastic.csa.iisc.ernet.in/~sindhupr/}{Sindhu Padakandla}, \href{https://sites.google.com/site/priyankpariharcs/}{Priyank Parihar} \&  \href{http://stochastic.csa.iisc.ernet.in/~chandru/}{Chandrashekar Lakshminarayanan} \\


\item\textsc{Human GAIT Analysis.} \\
\footnotesize{Human GAIT Analysis is the study of Human Motion and is used to figure out the problem in people with abnormal Gait and if possible fix it. The current gold standard method of doing a Gait Analysis is by using Optical system, which are very expensive and not so easy to use. We are looking into ways to use \textbf{IMU(Inertial Measurement Unit)} sensors, which comes with inbuilt Gyroscope, Accelerometer and Magnetometer, to perform Gait Analysis . We are currently looking at algorithms and methods which can make the results from IMU system clinically relevant. We can use Machine Learning techniques to provided suggestion to Doctors and Surgeons based on the data recorded.
}\\
\phantom{.}\hfill\small Advisors: Laura Rocchi, \href{http://drona.csa.iisc.ernet.in/~gopi/}{Prof. K. \textsc{Gopinath}}\\




\item \normalsize{\textsc{Modeling Storage Performance in a HPC System using Machine Learning.}}\\ \textbf{[ME Thesis]}\\
\footnotesize{We present a mathematical model that can capture the relationship between the \textit{features} (configuration parameters of a file system, hardware configuration and the workload configuration) and the \textit{performance metrics} (Read speed, write speed of disk etc.) and use this to rank the features according to their importance in deciding the performance of the parallel file system (Gluster FS). With the knowledge about the importance of the features and by using the prediction model, the bottleneck in the system can be identified which can help in improving the efficiency of the cluster.}\\\phantom{.}\hfill\small Advisor: \href{http://drona.csa.iisc.ernet.in/~gopi/}{Prof. K. \textsc{Gopinath}} | \normalsize
\textsc{Grade}: ``A`` (7/8) \\ \phantom{.}\hfill Thesis: \href{https://goo.gl/eglBjh}{https://goo.gl/eglBjh}\\
 \end{itemize}