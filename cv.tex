\documentclass[a4paper,10pt]{article}
\usepackage[a4paper,bindingoffset=0.2in,left=2.0cm,right=2.0cm,top=1.5cm,bottom=1.5cm,footskip=.25in]{geometry}

%A Few Useful Packages
\usepackage{marvosym}
\usepackage{fontspec} 					%for loading fonts
\usepackage{xunicode}
\usepackage{xltxtra}
\usepackage{url,parskip} 	%other packages for formatting
\RequirePackage{color,graphicx}
\usepackage[usenames,dvipsnames]{xcolor}
\usepackage[big]{layaureo} 				%better formatting of the A4 page
% an alternative to Layaureo can be ** \usepackage{fullpage} **
\usepackage{supertabular} 				%for Grades
\usepackage{titlesec}					%custom \section

%Setup hyperref package, and colours for links
\usepackage{hyperref}
\definecolor{linkcolour}{rgb}{0,0.2,0.6}
\hypersetup{colorlinks,breaklinks,urlcolor=linkcolour, linkcolor=linkcolour}

%FONTS
\defaultfontfeatures{Mapping=tex-text}
%\setmainfont[SmallCapsFont = Fontin SmallCaps]{Fontin.otf}

%%% modified for Karol Kozioł for ShareLaTeX use
\setmainfont[
SmallCapsFont = Fontin-SmallCaps.otf,
BoldFont = Fontin-Bold.otf,
ItalicFont = Fontin-Italic.otf
]
{Fontin.otf}
%%%

%CV Sections inspired by: 
%http://stefano.italians.nl/archives/26
\titleformat{\section}{\Large\scshape\raggedright}{}{0em}{}[\titlerule]
\titlespacing{\section}{0pt}{3pt}{3pt}
%Tweak a bit the top margin
%\addtolength{\voffset}{-1.3cm}

%Italian hyphenation for the word: ''corporations''
\hyphenation{im-pre-se}

%-------------WATERMARK TEST [**not part of a CV**]---------------
\usepackage[absolute]{textpos}

\setlength{\TPHorizModule}{30mm}
\setlength{\TPVertModule}{\TPHorizModule}
\textblockorigin{2mm}{0.65\paperheight}
\setlength{\parindent}{0pt}

%--------------------BEGIN DOCUMENT----------------------
\begin{document}

\pagestyle{empty} % non-numbered pages

%\font\fb=''[cmr10]'' %for use with \LaTeX command

%--------------------TITLE-------------
{\Huge Sandeep \textsc{Kumar}}\\\\
\begin{tabular}{ll}
    \textsc{DOB:} &  19 September 1989 \\
    \textsc{email:}     & \href{mailto:sandeep007734@gmail.com}{sandeep007734@gmail.com} \\
    \textsc{Home Page: }& \href{http://clweb.csa.iisc.ernet.in/sandeep.kumar/}{clweb.csa.iisc.ernet.in/sandeep.kumar/}
\end{tabular}
%--------------------SECTIONS-----------------------------------
%Section: Personal Data

%Section: Personal Data


%Section: Education
\section{Education}
\begin{tabular}{rl}	

 2011 - 2013 & Master of Engineering in \textsc{Computer Science}, 1$^{st} $ class\\
 &\normalsize\textbf{\href{http://www.csa.iisc.ernet.in/}{Computer Science and Automation Department,}}\\
  & Indian Institute of Science, Bangalore, India\\
& Thesis: \href{https://docs.google.com/viewer?a=v&pid=sites&srcid=ZGVmYXVsdGRvbWFpbnxzYW5kZWVwa3VtYXJ3ZWJzaXRlfGd4OjU2MWI0OWUzMjJmNjNjMmE}{``Modeling Storage Performance in a HPC System''}, Grade ``A'' (7/8) \\
& \small Advisor: \href{http://drona.csa.iisc.ernet.in/~gopi/}{Prof. K. \textsc{Gopinath}}\\\\
% | \normalsize \textsc{Grade}: ``A`` (7/8) \\
%&\normalsize \textsc{CGPA}: 5.9/8


2007 - 2011& Bachelor of Technology in \textsc{Computer Science}, 1$^{st} $ class\\
&\normalsize\textbf{\href{http://www.ipu.ac.in/}{Guru Gobind Singh Indraprastha Univeristy}}, New Delhi, India\\
&\normalsize \\
%\textsc{Result}: 73.5\%

%\textsc{March} 2006& Higher Secondary Examination, Class XII \\
%& \textbf{\href{http://kvsangathan.nic.in/}{Kendriya Vidyalya }}Vikas Puri, New Delhi, India\\
%&\textsc{Result}: 83.6\% \\&\\

\iffalse
\textsc{March} 2004& Secondary School Certificate, Class X \\
& \textbf{\href{http://kvsangathan.nic.in/}{Kendriya Vidyalya }}Vikas Puri, New Delhi, India\\
&\textsc{Result}: 83.6\% \\
\fi
\end{tabular}

\section{Research Interest}
Distributed and Parallel Systems, Operating Systems, Machine Learning.\\

\section{Research Projects}
\begin{itemize}



\item \normalsize{\textsc{Scalable Performance Tuning of Hadoop MapReduce: A Noisy Gradient Approach}} \\
\footnotesize{The simultaneous perturbation stochastic approximation (\textbf{SPSA}) algorithm is a noisy gradient algorithm that has been successfully deployed for parameter tuning in a variety of applications ranging from traffic control to service systems. The SPSA algorithm tunes the parameters by directly observing the performance of the real system. Further, the SPSA is independent of parameter dimensions and requires only two or fewer observations per iteration. We tune the Hadoop parameters using the Simultaneous Perturbation Stochastic Approximation (SPSA) algorithm in order to achieve better performance than the default configuration. We are currently comparing this method with other techniques such as Starfish.}\\\phantom{.}\hfill\small Advisor: \href{http://drona.csa.iisc.ernet.in/~gopi/}{Prof. K. \textsc{Gopinath}} \& \href{http://drona.csa.iisc.ernet.in/~shalabh/}{Prof. Shalabh Bhatnagar} 
\\\phantom{.}\hfill\small Collaborators: \href{http://stochastic.csa.iisc.ernet.in/~sindhupr/}{Sindhu Padakandla}, \href{https://sites.google.com/site/priyankpariharcs/}{Priyank Parihar} \&  \href{http://stochastic.csa.iisc.ernet.in/~chandru/}{Chandrashekar Lakshminarayanan} \\


\item\textsc{Human GAIT Analysis.} \\
\footnotesize{Human GAIT Analysis is the study of Human Motion and is used to figure out the problem in people with abnormal Gait and if possible fix it. The current gold standard method of doing a Gait Analysis is by using Optical system, which are very expensive and not so easy to use. We are looking into ways to use \textbf{IMU(Inertial Measurement Unit)} sensors, which comes with inbuilt Gyroscope, Accelerometer and Magnetometer, to perform Gait Analysis . We are currently looking at algorithms and methods which can make the results from IMU system clinically relevant. We can use Machine Learning techniques to provided suggestion to Doctors and Surgeons based on the data recorded.
}\\
\phantom{.}\hfill\small Advisors: Laura Rocchi, \href{http://drona.csa.iisc.ernet.in/~gopi/}{Prof. K. \textsc{Gopinath}}\\




\item \normalsize{\textsc{Modeling Storage Performance in a HPC System using Machine Learning.}}\\ \textbf{[ME Thesis]}\\
\footnotesize{We present a mathematical model that can capture the relationship between the \textit{features} (configuration parameters of a file system, hardware configuration and the workload configuration) and the \textit{performance metrics} (Read speed, write speed of disk etc.) and use this to rank the features according to their importance in deciding the performance of the parallel file system (Gluster FS). With the knowledge about the importance of the features and by using the prediction model, the bottleneck in the system can be identified which can help in improving the efficiency of the cluster.}\\\phantom{.}\hfill\small Advisor: \href{http://drona.csa.iisc.ernet.in/~gopi/}{Prof. K. \textsc{Gopinath}} | \normalsize
\textsc{Grade}: ``A`` (7/8) \\ \phantom{.}\hfill Thesis: \href{https://goo.gl/eglBjh}{https://goo.gl/eglBjh}\\
 \end{itemize}
 \newpage

\section{Publications}
\begin{itemize}
\item Sandeep Kumar, Sindhu Padakandla, Chandrashekar L, Priyank Parihar, K. G. (2017). Performance Tuning of Hadoop MapReduce: A Noisy Gradient Approach. IEEE Cloud, 8.\\ \href{https://arxiv.org/abs/1611.10052}{https://arxiv.org/abs/1611.10052}
\end{itemize}

 
\section{Course Projects}
 \begin{itemize}
\item\textsc{PINTOS: Operating System.} [2011]\\
\footnotesize{Worked on \textit{PINTOS} (a simple instructional Operating System) and implemented the functionality of Thread Synchronization, User Program, Virtual Memory and File System in the kernel of the Operating System.}\\\phantom{.}\hfill\small Advisor: \href{http://drona.csa.iisc.ernet.in/~gopi/}{Prof. K. \textsc{Gopinath}}

\item \normalsize{\textsc{Distributed Computing.}} [2012]\\
\footnotesize{Wrote \href{https://docs.google.com/viewer?a=v&pid=sites&srcid=ZGVmYXVsdGRvbWFpbnxzYW5kZWVwa3VtYXJ3ZWJzaXRlfGd4Ojc0OGQwYWEwZWYxMDgzMjE}{Distributed Programs} to solve TSP (Travelling sales man problem), ABP (Alpha Beta pruning search) and MST (Minimum spanning tree) using \textit{rpcgen} in C++ and showed a speed up of factor 9, 6 and 2.5 respectively when the number of servers went up from 1 to 6.}\hfill\small Advisor: \href{http://www.researchgate.net/profile/R_Hansdah/publications}{Prof. R.C. \textsc{Hansdah}}\\
\phantom{.}\hfill Report: \href{https://goo.gl/BnTpTF}{https://goo.gl/BnTpTF}

\item \normalsize{\textsc{Communication Network.}} [2012]\\
\footnotesize{Studied the algorithm \href{https://docs.google.com/viewer?a=v&pid=sites&srcid=ZGVmYXVsdGRvbWFpbnxzYW5kZWVwa3VtYXJ3ZWJzaXRlfGd4OjIyMWM1MDYxYjQ2MTVmNw}{SOFA (Sleep optimal Fair attention)}, which aims the energy conservation in wireless devices by changing the scheduling policy by simulating it to see the performance. }\\
\phantom{.}\hfill\small Advisor: \href{http://stochastic.csa.iisc.ernet.in/~shalabh/}{Prof. Shalabh \textsc{Bhatnagar}}\\ \phantom{.}\hfill Report: \href{https://goo.gl/Lh5QQ9}{https://goo.gl/Lh5QQ9}

\item \normalsize{\textsc{Game Theory.}} [2012]\\
\footnotesize{Studied existing scheme for handling \href{https://docs.google.com/viewer?a=v&pid=sites&srcid=ZGVmYXVsdGRvbWFpbnxzYW5kZWVwa3VtYXJ3ZWJzaXRlfGd4OjM4ZDhjYjhkYWJmM2Y0MDU}{Kidney Exchange Programs} and proposed a new scheme \textbf{SPAR} for this.}\\
\phantom{.}\hfill\small Advisor: \href{http://lcm.csa.iisc.ernet.in/hari/}{Prof. Y. \textsc{Narahari}}\\ \phantom{.}\hfill Report: \href{https://goo.gl/3Mujoh}{https://goo.gl/3Mujoh}

\item \normalsize{\textsc{Network and Distributed Systems Security.}} [2012]\\
\footnotesize{Design and implemented a Secure Email server, using \textit{Diffie Hellman} key exchange algorithm for secure exchange of keys and \textit{DES} algorithm for encrypting the messages using the keys exchanged earlier. \textit{SHA-512} hashing algorithm was used to store password on server side.}\hfill\small Advisor: \href{http://www.researchgate.net/profile/R_Hansdah/publications}{Prof. R.C. \textsc{Hansdah}}\\

\item \normalsize{\textsc{Software Architecture.}} [2012]\\
\footnotesize{Design and Implemented Hospital Insurance Portal using \textit{MVC (Model View Controller)} architecture for managing the insurance related activities of a Hospital and communicate with other services using \textit{REST} architecture. It had two Views, Desktop and Mobile, using the same Model and Controller. The Design was kept the design simple and flexible, so that it should be easy to upgrade.}\hfill\small Advisor: \href{https://www.linkedin.com/pub/raghu-hudli/3/582/ba4}{Prof. Raghu \textsc{Hudli}}\\

 \end{itemize}
\section{Work Experience}
\begin{tabular}{r|p{11cm}}
 \textsc{Sept 14-Present} & \textsc{Indian Institute of Science}, Bangalore, Karnataka \\
 &\emph{Research Associate}\\\\
% &\footnotesize{Details about the work in the Research Project Section.}\\\multicolumn{2}{c}{} \\

 \textsc{Jul 13-Jun 14} & \textsc{Dell R\&D}, Bangalore, India \\&\emph{Software Development Engineer}\\&\footnotesize{At Dell R\&D, I was a part of the BizClient team working on BIOS configuration and system management tools, \textit{DCC} (Dell Command Configure) and \textit{OMCI} (Open Manage Client Instrumentation) respectively. DCC allows BIOS configuration from the Desktop (Windows and Linux) and OMCI allows remote management application programs to access information about the client computer. 
My primary job was to implement new features in the program as they become available in the BIOS. I also worked on remapping the GUI of the DCC using WPF (Windows Presentation Foundation).}\\\multicolumn{2}{c}{} \\
 
% \textsc{Jun 10-Jul 2010} & \textsc{NDMC, IT Department}, New Delhi, India \\
% &\emph{Intern}\\
% &\footnotesize{At NDMC (New Delhi Municipal Corporation), IT department, I was part of the team that worked on building a \textit{Security Surveillance System} using IP (Internet Protocol) camera, mounted at traffic signals and in various others offices. I worked on Motion detecting algorithms, used to optimize the uses of such cameras. The program was implemented C\#.}\\\multicolumn{2}{c}{} \\
\end{tabular}


%%Section: Scholarships and additional info
%\section{Scholarships and Certificates}
%\begin{tabular}{rl}
% \textsc{Mar.} 2011 & 642 Rank in GATE (Graduate Aptitude Test in Engineering) Exam 2011 (Total Students: 136027) \\
%2004  & Scholarship for Scoring above than 80\% in class X
%\end{tabular}

%Section: Scholarships and additional info
%\section{Training and Conferences}
%\begin{tabular}{rl}
% Present & Various Meetups on Docker and GlusterFS \\
% \textsc{May-Jun} 2012 & Microsoft Summer School on Distributed Algorithms and System IISc \\
% \textsc{Jun-Jul} 2010 & Attended training on .Net Platform at Ambedkar institute of Technology \\
%  \textsc{Jun-Jul} 2009 & Completed training in Java SE6 at NIIT 
%\end{tabular}
%\\
\section{Talk}
\begin{itemize}
\item \normalsize{\textsc{Using IMU sensors and Android for Human Gait Analysis}} \\
\footnotesize{Gave a talk on techniques used for doing Human Gait Analysis using IMU (Inertial Measurement Sensors) and Android smart phone along with some preliminary results at Robert Center for Cyber Physical System, IISc, Bangalore.
}\\\phantom{}\hfill Slides: \href{https://goo.gl/x6XRWw}{https://goo.gl/x6XRWw}
\end{itemize}

\section{Technical Skills}
\begin{itemize}
\item Programming:
\begin{itemize}
\item Proficient in C++, C, JAVA, .NET (C\# and WPF).
\item Prior Experience in R, Python, MATLAB, SQL, JavaScript.
\end{itemize}
\item Working knowledge of a \textit{Linux Kernel}.
\item Experience with \textit{Hadoop} and \textit{GlusterFS}.
\item Experience in Developing \textbf{Android Apps} and maintaining the Back-end Server.
\begin{itemize}
\item \href{ https://play.google.com/store/apps/details?id=com.sportsjyotishi.cricketjyotishi}{Cricket Jyotishi}
\item \href{ https://play.google.com/store/apps/details?id=app.skdon.dailyquote}{Daily Book Quotes}
\item Human Gait Analysis (In Progress)
\end{itemize}

\end{itemize}

\section{References}

Gopinath K.\\
Professor\\
\href{mailto:gopi@csa.iisc.ernet.in}{gopi@csa.iisc.ernet.in}\\
Computer Science and Automation\\
Indian Institute of Science.\\

%%\iffalse
%\begin{tabular}{  p{\dimexpr 0.5\linewidth-2\tabcolsep} 
%                   p{\dimexpr 0.5\linewidth-2\tabcolsep} }
%K.Gopinath & R C Hansdah \\
%Professor  & Associate Professor \\
%\href{mailto:gopi@csa.iisc.ernet.in}{gopi@csa.iisc.ernet.in} & \href{mailto:hansdah@csa.iisc.ernet.in}{hansdah@csa.iisc.ernet.in}\\
%Computer Science and Automation & Computer Science and Automation \\
%Indian Institute of Science & Indian Institute of Science \\
%\end{tabular}  
%
%
%\begin{tabular}{  p{\dimexpr 0.5\linewidth-2\tabcolsep} 
%                   p{\dimexpr 0.5\linewidth-2\tabcolsep} }
%Sanjit Chatterjee & Raghu Hudli  \\Human GAIT Analysis
%Assistant Professor  & CEO ObjectOrb Technologies Pvt. Ltd. \\
%\href{mailto:sanjit@csa.iisc.ernet.in}{sanjit@csa.iisc.ernet.in}  & \href{mailto:raghu@hudli.com}{raghu@hudli.com } \\
%Computer Science and Automation & Visting Faculty  \\
%Indian Institute of Science & Computer Science and Automation \\
% & Indian Institute of Science \\
%\end{tabular}  
%%\fi

\section{Interests and Activities}
\begin{itemize}
\item Apart from field of Computer Science, I love to read books, specially Fiction, Biographies and History books.\\ List of Books read so far :\href{https://goo.gl/bEjjJJ}{https://goo.gl/bEjjJJ}
\item I love to play an Online strategy game, called Defense of the Ancients 2 (DOTA 2).\\
Profile: \href{http://www.dotabuff.com/players/88064784}{http://www.dotabuff.com/players/88064784}
\item I occasionally  go for Cycling and Trekking Trips.\\
Some Pics: \href{https://plus.google.com/photos/107120438417592006252/albums/6077721099605222449?authkey=CKLDmKXmzrDFkgE}{https://goo.gl/ue6qeH} \\
Strava Profile: \href{https://goo.gl/F1ow46}{https://goo.gl/F1ow46}
\end{itemize}
\end{document}
